\documentclass{sigchi}

% Use this section to set the ACM copyright statement (e.g. for
% preprints).  Consult the conference website for the camera-ready
% copyright statement.

% Copyright
\CopyrightYear{2016}
%\setcopyright{acmcopyright}
\setcopyright{acmlicensed}
%\setcopyright{rightsretained}
%\setcopyright{usgov}
%\setcopyright{usgovmixed}
%\setcopyright{cagov}
%\setcopyright{cagovmixed}
% DOI
\doi{http://dx.doi.org/10.475/123_4}
% ISBN
\isbn{123-4567-24-567/08/06}
%Conference
\conferenceinfo{CHI'16,}{May 07--12, 2016, San Jose, CA, USA}
%Price
\acmPrice{\$15.00}

% Use this command to override the default ACM copyright statement
% (e.g. for preprints).  Consult the conference website for the
% camera-ready copyright statement.

%% HOW TO OVERRIDE THE DEFAULT COPYRIGHT STRIP --
%% Please note you need to make sure the copy for your specific
%% license is used here!
% \toappear{
% Permission to make digital or hard copies of all or part of this work
% for personal or classroom use is granted without fee provided that
% copies are not made or distributed for profit or commercial advantage
% and that copies bear this notice and the full citation on the first
% page. Copyrights for components of this work owned by others than ACM
% must be honored. Abstracting with credit is permitted. To copy
% otherwise, or republish, to post on servers or to redistribute to
% lists, requires prior specific permission and/or a fee. Request
% permissions from \href{mailto:Permissions@acm.org}{Permissions@acm.org}. \\
% \emph{CHI '16},  May 07--12, 2016, San Jose, CA, USA \\
% ACM xxx-x-xxxx-xxxx-x/xx/xx\ldots \$15.00 \\
% DOI: \url{http://dx.doi.org/xx.xxxx/xxxxxxx.xxxxxxx}
% }

% Arabic page numbers for submission.  Remove this line to eliminate
% page numbers for the camera ready copy
% \pagenumbering{arabic}

% Load basic packages
\usepackage{balance}       % to better equalize the last page
\usepackage{graphics}      % for EPS, load graphicx instead 
\usepackage[T1]{fontenc}   % for umlauts and other diaeresis
\usepackage{txfonts}
\usepackage{mathptmx}
\usepackage[pdflang={en-US},pdftex]{hyperref}
\usepackage{color}
\usepackage{booktabs}
\usepackage{textcomp}

% Some optional stuff you might like/need.
\usepackage{microtype}        % Improved Tracking and Kerning
% \usepackage[all]{hypcap}    % Fixes bug in hyperref caption linking
\usepackage{ccicons}          % Cite your images correctly!
% \usepackage[utf8]{inputenc} % for a UTF8 editor only

% If you want to use todo notes, marginpars etc. during creation of
% your draft document, you have to enable the "chi_draft" option for
% the document class. To do this, change the very first line to:
% "\documentclass[chi_draft]{sigchi}". You can then place todo notes
% by using the "\todo{...}"  command. Make sure to disable the draft
% option again before submitting your final document.
\usepackage{todonotes}

% Paper metadata (use plain text, for PDF inclusion and later
% re-using, if desired).  Use \emtpyauthor when submitting for review
% so you remain anonymous.
\def\plaintitle{Prediction of Walking One Step Earlier \\ for Realtime Walking Guidance}
\def\plainauthor{First Author, Second Author, Third Author,
  Fourth Author, Fifth Author, Sixth Author}
\def\emptyauthor{}
\def\plainkeywords{Authors' choice; of terms; separated; by
  semicolons; include commas, within terms only; required.}
\def\plaingeneralterms{Documentation, Standardization}

% llt: Define a global style for URLs, rather that the default one
\makeatletter
\def\url@leostyle{%
  \@ifundefined{selectfont}{
    \def\UrlFont{\sf}
  }{
    \def\UrlFont{\small\bf\ttfamily}
  }}
\makeatother
\urlstyle{leo}

% To make various LaTeX processors do the right thing with page size.
\def\pprw{8.5in}
\def\pprh{11in}
\special{papersize=\pprw,\pprh}
\setlength{\paperwidth}{\pprw}
\setlength{\paperheight}{\pprh}
\setlength{\pdfpagewidth}{\pprw}
\setlength{\pdfpageheight}{\pprh}

% Make sure hyperref comes last of your loaded packages, to give it a
% fighting chance of not being over-written, since its job is to
% redefine many LaTeX commands.
\definecolor{linkColor}{RGB}{6,125,233}
\hypersetup{%
  pdftitle={\plaintitle},
% Use \plainauthor for final version.
%  pdfauthor={\plainauthor},
  pdfauthor={\emptyauthor},
  pdfkeywords={\plainkeywords},
  pdfdisplaydoctitle=true, % For Accessibility
  bookmarksnumbered,
  pdfstartview={FitH},
  colorlinks,
  citecolor=black,
  filecolor=black,
  linkcolor=black,
  urlcolor=linkColor,
  breaklinks=true,
  hypertexnames=false
}

% create a shortcut to typeset table headings
% \newcommand\tabhead[1]{\small\textbf{#1}}

% End of preamble. Here it comes the document.
\begin{document}

\title{\plaintitle}

\numberofauthors{3}
\author{%
  \alignauthor{Leave Authors Anonymous\\
    \affaddr{for Submission}\\
    \affaddr{City, Country}\\
    \email{e-mail address}}\\
  \alignauthor{Leave Authors Anonymous\\
    \affaddr{for Submission}\\
    \affaddr{City, Country}\\
    \email{e-mail address}}\\
  \alignauthor{Leave Authors Anonymous\\
    \affaddr{for Submission}\\
    \affaddr{City, Country}\\
    \email{e-mail address}}\\
}

\maketitle

\begin{abstract}
  UPDATED---\today. This sample paper describes the formatting
  requirements for SIGCHI conference proceedings, and offers
  recommendations on writing for the worldwide SIGCHI
  readership. Please review this document even if you have submitted
  to SIGCHI conferences before, as some format details have changed
  relative to previous years. Abstracts should be about 150 words and
  are required.
  
%%
ヒトの歩行運動は,突発的な感覚入力への姿勢反射応答の潜時が約2歩であることが知られている.
そのため歩行を実時間で誘導するためには,現在の歩行運動状態から2歩先行した歩行運動を実時間で
予測し続ける必要がある.そこで本研究では多層パーセプトロンを用いた実時間歩行運動予測を行い,
直進・右左折を含む5種の歩行運動に対し,1歩先行した歩行予測を実現したので報告する.

\end{abstract}

\category{H.5.m.}{Information Interfaces and Presentation
  (e.g. HCI)}{Miscellaneous} \category{See
  \url{http://acm.org/about/class/1998/} for the full list of ACM
  classifiers. This section is required.}{}{}

\keywords{\plainkeywords}

\section{Introduction}

%%

ヒトの歩行運動は,突発的な感覚入力への姿勢反射応答の潜時が約2歩であることが知られている\cite{bib01}.
そのため歩行を実時間で誘導するためには,現在の歩行運動状態から2歩先行した歩行運動を実時間で
予測し続ける必要がある.そこで本研究では多層パーセプトロンを用いた実時間歩行運動予測を行い,
直進・右左折を含む5種の歩行運動に対し,1歩先行した歩行予測を実現したので報告する.

本研究の目的は,歩行の予測可能性を示すことにあるため,
任意の歩行運動を対象とするのではなく,予め明示的に設計された経路を歩行した際の
歩行運動を対象とする.そこで本研究の問題を
「現在時刻の歩行運動から$\Delta t$秒先の歩行運動をどの程度予測可能か」
と設定する.ここで本研究の主目的は経路案内を達成するための歩行誘導であるため,
予測対象である「歩行運動」は頭部の移動軌跡が得られれば十分であると考える.

本研究では連続運動としての相関が現時刻・先行時刻間で低下したとしても,
先行時刻における運動を推定するに足る情報量が現在時刻に内包されているのならば,
推定精度が十分高いことが期待できる先行時間幅が存在すると仮定している.
従って予測対象は物理量を有する身体運動であるべきであって,
右折,左折といった名義尺度ではない.これは歩行予測と誘導を実時間で
実現するためには,物理量としての身体運動が求められるからである.

This format is to be used for submissions that are published in the
conference proceedings. We wish to give this volume a consistent,
high-quality appearance. We therefore ask that authors follow some
simple guidelines. You should format your paper exactly like this
document. The easiest way to do this is to replace the content with
your own material.  This document describes how to prepare your
submissions using \LaTeX.

\section{Page Size and Columns}
On each page your material should fit within a rectangle of 7 $\times$
9.15 inches (18 $\times$ 23.2 cm), centered on a US Letter page (8.5
$\times$ 11 inches), beginning 0.85 inches (1.9 cm) from the top of
the page, with a 0.3 inches (0.85 cm) space between two 3.35 inches
(8.4 cm) columns. Right margins should be justified, not
ragged. Please be sure your document and PDF are US letter and not A4.

\section{Typeset Text}
The styles contained in this document have been modified from the
default styles to reflect ACM formatting conventions. For example,
content paragraphs like this one are formatted using the Normal style.

\LaTeX\ sometimes will create overfull lines that extend into columns.
To attempt to combat this, the \texttt{.cls} file has a command,
\texttt{{\textbackslash}sloppy}, that essentially asks \LaTeX\ to
prefer underfull lines with extra whitespace.  For more details on
this, and info on how to control it more finely, check out
{\url{http://www.economics.utoronto.ca/osborne/latex/PMAKEUP.HTM}}.
%%
\section{Modeling 歩行運動計測と歩行運動のモデル化}%%%%%%%%%%%%%%%%%%%%

\subsection{Measure Walking Behavior歩行運動の計測}
\label{subsec-nac}



まず取り扱う物理量の選定であるが,本稿では歩行運動を表現する物理量として速度を用いている.
これは歩行運動は連続した周期運動であるためである.
つまり時刻$t$において観測される身体各部位の速度は,
周期$T$後に再び観測されるという仮説が成立しうる利点がある.
さらにこの仮説は歩行位置によらず成立するため,速度という物理的特徴量は,
空間内の任意の位置を移動する歩行者を対象とした実時間歩行運動予測に適した特徴を与える.

そこで歩行運動を計測するためのマーカ配置として,図\ref{fig-marker}に示したHelen Hayers配置を用いた.
nac社製Kestrel(220万画素)を8台用いて300fpsにて撮影を行なった.25点の座標$\{p_x,p_y,p_z\}$を
計測したデータはMAC3D System Cortexを用いて,欠損データの補間処理,
カットオフ周波数6Hzのローパスフィルタ処理,
後退差分法による25点マーカの速度ベクトル$\{v_x,v_y,v_z\}$を求めた.
\begin{figure}[tb]
  \begin{center}
  \vspace*{-6mm}
    \includegraphics*[width=30mm]{fig/marker.eps}
  \end{center}
  \vspace*{-6mm}
  \caption{マーカ配置(HelenHayers)}
  \label{fig-marker}
\end{figure}




\subsection{歩行条件}

定量的に予測性能を評価するために,
経路条件として「直進,90度右折,90度左折,45度右折,45度左折」の5条件を設定した.
歩行距離は1試行につき6[m]であり,その内訳は始点から3mの直進区間を経て
右左折もしくは直進区間としてさらに3[m]である.
5条件の歩行経路を明示するため床に印を貼り,
通常の歩行速度や歩幅で男性被験者1名に歩行させた.

計測データの中から学習データセットとして用いた頭部軌跡を図\ref{fig-training-data}に示す.
同図は図\ref{fig-marker}中の黒矢印として示した頭頂部の軌跡であり,5条件を各2回分の歩行データを含む.
経路5条件に従って,水平面内の原点付近で右左折をしている様子が確認できる.
垂直面内の脈動は頭部の上下動を表している.
第\ref{subsec-nac}節の通り,1試行のデータは各時刻あたり全25点分のマーカの
速度ベクトル(3次元)の計75点が300[fps]で記録された時系列データである.

\begin{figure}[tb]
  \begin{center}
    \includegraphics*[width=80mm]{fig/TRAINING-DATA.eps}
  \end{center}
  \vspace*{-10mm}
  \caption{学習データ群の頭頂部マーカ軌跡(図\ref{fig-marker}黒矢印)}
  \label{fig-training-data}
\end{figure}




\subsection{入出力データ群のモデル化}
\label{projection}

推定対象データ群(入力データ)・被推定対象データ群(出力データ)を用いて歩行運動をモデル化する.

今時刻$t$におけるマーカ$m$の速度を$\check{v}_{m,t} $と定義し,
全マーカの集合を$X(t) \in \mathbb{R}^{1 \times 75}$,頭頂部マーカのみの集合を$Y(t) \in \mathbb{R}^{1 \times 3}$と定義する.
また現在時刻から予測対象の先行時刻までの幅を「先行時間幅$\Delta t$」と表す.
ここで本研究の問題は次のように定式化される.すなわち,
$\Delta t = 0 $[s] ならば,「現在の歩行運動から現在の頭頂部運動の予測」問題を解けば良い.
これは写像$f_{\Delta t = 0}: X(t) \rightarrow Y(t)$を推定するモデル同定と等価である.
同様に$\Delta t = 0.5 $[s] ならば「現在の歩行運動から0.5[s]先の頭頂部運動の予測」となる
写像$f_{\Delta t = 0.5}: X(t) \rightarrow Y(t+\Delta t)$の同定問題に帰着される.

期待される予測精度は,先行時間幅が増加するに伴い低下すると考えられる.
これは連続周期運動である歩行の運動特徴量としての相関が,
そこで本稿では計測したデータから求めた2歩分の時間1.1[s]を基準歩行周期とみなし,
同条件として$\Delta t = \{0.0, \  0.5, \ 1.1\}$ [s]の3条件を設定した.
これは歩数表現でそれぞれ${0, 1, 2}$歩先の予測に対応する.
予測精度は$\Delta t = 0$ が最も高く,$\Delta t = 1.1$ が最も低いと推定される.
しかし推定精度が低くとも,例えば,右左折の判断に足る運動予測の可能性を明らかにすることに本稿の主眼がある.


\section{予測器設計と歩行運動学習}%%%%%%%%%%%%%%%%%%%%

\subsection{予測器設計}


予測器は図\ref{fig-dbn}に示すDeep belief networkを用いた.
細胞数は入力層,隠れ層,出力層それぞれ75,100,3である.
入力層に現在時刻$t$における25点分のマーカの速度ベクトル($X(t) \in \mathbb{R}^{1 \times 75}$)を与え,
出力層に先行時刻幅$\Delta t$先の頭頂部マーカの速度ベクトル($Y(t + \Delta t) \in \mathbb{R}^{1 \times 3}$)を推定させる.
活性化関数はシグモイド関数(sigm),出力関数は恒等写像関数(linear)である.


\begin{figure}[tb]
  \begin{center}
    \includegraphics*[width=60mm]{fig/DBN_3layers.eps}
  \end{center}
  \vspace*{-6mm}
  \caption{Deep belief network}
  \label{fig-dbn}
\end{figure}



\subsection{正規化}

学習のための前処理として,入出力データの正規化を行う.

まず第\ref{projection}節で定義した写像則に従い教師データ群を作成し,
$X(t) \in \mathbb{R}^{1 \times 75}, Y(t + \Delta t) \in \mathbb{R}^{1 \times 3}$をそれぞれ得る.
これらの集合は速度の実測値 $\check{v}_{m,t}$ を持つ.ここで$m$はマーカやデカルト座標系$\{ x, y, z\}$に区別なく付与される系列番号であり,
$X$,$Y$に対しそれぞれ$1 \leq m_X \leq 75$,$1 \leq m_Y \leq 3$を与える.
正規化とは,全時刻域$(1 \leq t \leq N)$で系列$m$の集合の平均を 0 標準偏差を1に変換することである.

そこでまず系列$m$内平均$\overline{\check{v}_{m}} $を式(\ref{eq.average})で求める.
ただし$C$は全条件数であり,$n_c$は各条件$c$が含む時間ステップ数である.

% m : marker index number
% Tn = all seconds
% average 
\begin{equation}
\overline{\check{v}_{m}} = \frac{1}{N}\sum^{N}_{t=1}\check{v}_{m,t} 
 \hspace{3mm} where  \hspace{3mm}
N=\sum^{C}_{c=1} n_c 
\label{eq.average}
\end{equation}

次に,系列$m$内標準偏差$\check{s}_m$を式(\ref{eq.standardDeviation})で求める.

% standard deviation
\begin{equation}
  \check{s}_m = \sqrt{ \frac{1}{N-1}\sum^{N}_{t=1} ( \check{v}_{m,t} - \overline{\check{v}_m} ) ^2} 			
\label{eq.standardDeviation}
\end{equation}

最後に,系列$m$は式(\ref{eq.regularization})により正規化された$ v_{m,t} $を得る.

% regularization
\begin{equation}
  v_{m,t} = \frac{\check{v}_{m,t} - \overline{\check{v}_{m}}}{\check{s}_m} 
  \label{eq.regularization}
\end{equation}


\begin{figure*}[tb]
  \begin{center}
    \includegraphics*[width=180mm]{fig/velocity.eps}
  \end{center}
  \vspace*{-10mm}
  \caption{$x$軸方向の速度予測結果($\Delta t$は先行時間幅を表す)}
  \label{fig-velocity-result}
\end{figure*}

\subsection{予測実験}
現在時刻の速度入力マーカ25点$X(t) \in \mathbb{R}^{1 \times 75}$に対する
頭頂部速度$Y(t + \Delta t) \in \mathbb{R}^{1 \times 3}$を前述の通り正規化し教師データとして学習させた.
教師データ長は,先行時間幅$\Delta  t =\{0.0, \ 0.5,\  1.1\} $[s]ごとにそれぞれ12976, 11476,  9676である.
バッチサイズは各先行時間幅ごとにそれぞれ12976, 11476, 9676とし,学習回数を 2000回とし学習させた.
学習時の各種パラメータは脚注
\footnote{Learning rate  4, momentum 0.5, dropoutFraction 0.5, 
inputZeroMaskedFraction 0.2, weightMaxL2norm 15}に示す.
以上の手続きから,各先行時間幅3条件における予測器$f_{\Delta t = \{0.0, 0.5, 1.1\}}$をそれぞれ3つ得た.

評価用データ群として,教師データ群とは異なるデータ群を用いた.
ただし評価用データ群も同一被験者から教師データ群と同一日に計測したものである.


%
%見出しのレベルは3段階とし,
%第1レベル(章)は,上に1行あけて{\bf ゴシック体10 pt}により
%「{\bf 3. 章の見出し}」のように記入してください(\verb+\section+を使用).
%\subsubsection{項の見出し}
%
%第3レベル(項)の見出しも前後に空白行を設けず,
%{\bf ゴシック体9 pt}により「{\bf 3.2.1 項の見出し}」のように記入してください(\verb+\subsubsection+を使用).


%%% この\newpageは,最終ページの左右カラム高さを手動で調整する場合に挿入します.
%%% balanceパッケージを使用すれば自動的に左右の高さを揃えられます.
%\newpage

\section{予測結果}%%%%%%%%%%%%%%%%%%%%

\subsection{予測された速度の時系列評価}

実験結果を図\ref{fig-velocity-result}に示す.
同図は横軸を時間,縦軸を歩行者の初期位置における左右方向の頭頂部の速度成分とした時系列推定結果を示している.
同図は世界座標系で表現された速度であるため,$\Delta t = 0.0$[s]の $t=7$[s]付近以降のように大きく負の値を
とる場合は,歩行者は右折したことを表している.

図\ref{fig-velocity-result}のいずれのプロットも評価用データ群に関するものであり,
青い一点鎖線のGround Truth(期待する結果)が$Y(t+\Delta t)$,
太い橙色のPredicted(予測された結果)が$f_{\Delta t}(X(t))$,
細い橙色のCurrent Velocity(入力層に与えられた現時刻での速度)が$X(t)$をそれぞれ示している.
予測器の同定に十分成功していれば予測結果$f_{\Delta t}(X(t))$ が期待する結果$Y(t + \Delta t)$に漸近する.

まず図\ref{fig-velocity-result}上段の先行時刻幅$\Delta t= 0.0$[s]を観察する.
先行時間幅の定義より$\Delta t= 0.0$から$X(t)=Y(t)$のため,プロットGround TruthはCurrent Velocityと一致している.
予測された結果$f_{0.0}(X(t))$が$X(t)$に漸近していれば予測器の獲得が失敗していないことを示す
コントロールデータとなりうる.図\ref{fig-velocity-result}上段から,$f_{0.0}(X(t))$(プロットPredicted)は
概ね$Y(t)$(プロットGround Truth)と一致しており,期待する結果であるといえる.

次に図\ref{fig-velocity-result}中段の先行時刻幅$\Delta t= 0.5$[s]を観察する.
同図の$X(t)$が$Y(t+0.5)$と位相差を約$\pi$もつことから,
1歩先の運動予測のための評価用データ群であることが確認できる.
同様に$f_{0.5}(X(t))$と$Y(t+0.5)$を比較すると,
図\ref{fig-velocity-result}中段に黒矢印で示した時刻付近で右左折を意味する速度変化が
Ground Truthに沿って予測されることがわかった.
これは図中の$f_{0.5}(X(t))$(プロットPredicted)は$X(t)$の速度と反して$Y(t+0.5)$に漸近した
変動を示していることから,1歩先の頭頂部速度予測可能性を示すものであるといえる.

最後に図\ref{fig-velocity-result}下段の先行時刻幅$\Delta t= 1.1$[s]を観察する.
同様に$X(t)$が$Y(t+1.1)$と位相差を約$2\pi$もつことから,2歩先の評価用データ群であることが確認できる.
さらに$f_{1.1}(X(t))$と$Y(t+1.1)$を比較すると,$f_{1.1}(X(t))$のデータが右左折を意味する速度変動を
予測すべき時刻から大きく遅れて予測されていることがわかった.
すなわち,2歩先の頭頂部速度予測可能性は今回用いたデータセット群では
確認することができなかった.


\subsection{歩行軌跡の推定と評価}

時刻$t$における頭頂部速度の推定結果は前述の通り正規化されていることから,
式(\ref{eq.average})と式(\ref{eq.standardDeviation})で求めた
系列$m$内平均$\overline{\check{v}_{m}} $と標準偏差$\check{s}_m$を用いた逆変換を行い
実世界座標系での速度ベクトルを導出する.そこで
式(\ref{eq.integration})により初期座標を基準とした時間積分を行い頭頂部軌跡を求めた.
ただし$\delta t = 1/300 $[fps] である.

\begin{equation}
\vec{p}(t) = \sum^{t}_{t=1}(\check{s}_m  v_{m,t} +  \overline{\check{v}_{m}})  \delta t+ \vec{p}_0(t_0)
\label{eq.integration}
\end{equation}


図\ref{fig-0.0sec}〜図\ref{fig-1.1sec}はそれぞれ先行時間幅$\Delta  t =\{0.0, \ 0.5,\  1.1\} $[s]ごとに
図\ref{fig-velocity-result}の予測された時系列速度(プロットPredicted)を
式(\ref{eq.integration})を用いて頭頂部軌跡としてプロットした結果の天面図である.
細実線が$Y(t+\Delta t)$の期待する結果であり,太実線が$f_{\Delta t}(X(t))$の予測結果を示している.
歩行者は図の左側から右側に向かって歩行を開始し原点付近で五分岐していることから,
各歩行軌跡5条件に従った歩行した様子が観察できる.

図\ref{fig-0.0sec}〜図\ref{fig-1.1sec}から,先行時刻幅が増加するに従い,
期待する頭部軌跡から予測された軌跡が逸脱していく様子が観察された.
$\Delta t = 0.0$[s]条件である図\ref{fig-0.0sec}は期待する結果と予測された軌跡は一致しており,
$\Delta t = 0.5$[s]条件である図\ref{fig-0.5sec}は最大約500[mm]程度の位置誤差が確認された.また
$\Delta t = 1.1$[s]条件である図\ref{fig-1.1sec}は,右左折条件で特に
概ね2歩分に相当する約1000[mm]の位置誤差が原点付近で観察された.
これは前節で述べた通り予測された速度変動の時刻が期待する時刻で観測されなかったことに起因する.


\begin{figure}[tb]
  \begin{center}
    \includegraphics*[width=60mm]{fig/shift_0.0sec/trail.eps}
  \end{center}
  \vspace*{-10mm}
  \caption{$\Delta t$ = 0.0[s]}
  \label{fig-0.0sec}
%\end{figure}

%\begin{figure}[b]
  \begin{center}
      \includegraphics*[width=60mm]{fig/shift_0.5sec/trail.eps}
  \end{center}
  \vspace*{-10mm}
  \caption{$\Delta t$ = 0.5[s]}
  \label{fig-0.5sec}
%\end{figure}


%\begin{figure}[tb]
  \begin{center}
    \includegraphics*[width=60mm]{fig/shift_1.1sec/trail.eps}
  \end{center}
  \vspace*{-10mm}
  \caption{$\Delta t$ = 1.1[s]}
  \label{fig-1.1sec}
\end{figure}


%
%
%
%参考文献は出現順に番号を付け,該当個所に\cite{bib01}\cite{bib02}\cite{bib03}\cite{bib04}のようにカギカッコで指示してください.
%参考文献の引用リストは例を参考にして,文末に1行あけ,
%{\normalsize\bf ゴシック体10 pt}センタリングで「{\normalsize\bf 参考文献}」と記した後に,番号順に記入してください.
%姓名の記法や誌名巻号の略記法など形式について厳密な指定はありませんが,リストの中で統一を取るようにしてください.
%
%\section{PDF出力}%%%%%%%%%%%%%%%%%%%%
%原稿はPDFにて投稿してください.投稿前に,
%\begin{itemize}
% \item 書式の乱れがないか
% \item 参照文献番号があっているか
% \item 画像は印刷に耐えうるクオリティか
% \item A4サイズになっているか
% \item 白紙ページがないか
% \item 規定ページ数に収まっているか
%\end{itemize}
%などを必ず確認してください.
%また,出版プロセスでPDFを加工(ノンブル付与)するため,PDFのセキュリティなどは解除いただく様お願いいたします.




%%% balance.styを使用する場合,最後の\sectionの直前に入れます.
\balance

\section{むすび}%%%%%%%%%%%%%%%%%%%%
ヒトの歩行運動は,突発的な感覚入力への姿勢反射応答の潜時が約2歩であることが知られている.
そのため歩行を実時間で誘導するためには,現在の歩行運動状態から2歩先行した歩行運動を実時間で
予測し続ける必要がある.そこで本研究では多層パーセプトロンを用いた実時間歩行運動予測を行い,
直進・右左折を含む5種の歩行運動に対し,1歩先行した歩行予測を実現した.

今後の課題として現在は被験者数が1名に限られており教師データ数が少ない.
また本報告は訂正的な評価が主体となっており,定量的な評価が求められる.
さらに入出力層設計の妥当性に対する評価を行う必要がある.

%%%%%%%%%%%%%%%%%%%%%%

\subsection{Title and Authors}

Your paper's title, authors and affiliations should run across the
full width of the page in a single column 17.8 cm (7 in.) wide.  The
title should be in Helvetica or Arial 18-point bold.  Authors' names
should be in Times New Roman or Times Roman 12-point bold, and
affiliations in 12-point regular.  

See \texttt{{\textbackslash}author} section of this template for
instructions on how to format the authors. For more than three
authors, you may have to place some address information in a footnote,
or in a named section at the end of your paper. Names may optionally
be placed in a single centered row instead of at the top of each
column. Leave one 10-point line of white space below the last line of
affiliations.

\subsection{Abstract and Keywords}

Every submission should begin with an abstract of about 150 words,
followed by a set of Author Keywords and ACM Classification
Keywords. The abstract and keywords should be placed in the left
column of the first page under the left half of the title. The
abstract should be a concise statement of the problem, approach, and
conclusions of the work described. It should clearly state the paper's
contribution to the field of HCI\@.

\subsection{Normal or Body Text}

Please use a 10-point Times New Roman or Times Roman font or, if this
is unavailable, another proportional font with serifs, as close as
possible in appearance to Times Roman 10-point. Other than Helvetica
or Arial headings, please use sans-serif or non-proportional fonts
only for special purposes, such as source code text.

\subsection{First Page Copyright Notice}
This template include a sample ACM copyright notice at the bottom of
page 1, column 1.  Upon acceptance, you will be provided with the
appropriate copyright statement and unique DOI string for publication.
Accepted papers will be distributed in the conference
publications. They will also be placed in the ACM Digital Library,
where they will remain accessible to thousands of researchers and
practitioners worldwide. See
\url{http://acm.org/publications/policies/copyright_policy} for the
ACM's copyright and permissions policy.

\subsection{Subsequent Pages}

On pages beyond the first, start at the top of the page and continue
in double-column format.  The two columns on the last page should be
of equal length.

\begin{figure}
\centering
  \includegraphics[width=0.9\columnwidth]{figures/sigchi-logo}
  \caption{Insert a caption below each figure. Do not alter the
    Caption style.  One-line captions should be centered; multi-line
    should be justified. }~\label{fig:figure1}
\end{figure}

\subsection{References and Citations}

Use a numbered list of references at the end of the article, ordered
alphabetically by last name of first author, and referenced by numbers
in
brackets~\cite{acm_categories,ethics,Klemmer:2002:WSC:503376.503378}.
Your references should be published materials accessible to the
public. Internal technical reports may be cited only if they are
easily accessible (i.e., you provide the address for obtaining the
report within your citation) and may be obtained by any reader for a
nominal fee. Proprietary information may not be cited. Private
communications should be acknowledged in the main text, not referenced
(e.g., ``[Borriello, personal communication]'').

References should be in ACM citation format:
\url{http://acm.org/publications/submissions/latex_style}. This
includes citations to internet
resources~\cite{acm_categories,cavender:writing,CHINOSAUR:venue,psy:gangnam}
according to ACM format, although it is often appropriate to include
URLs directly in the text, as above.


% Use a numbered list of references at the end of the article, ordered
% alphabetically by first author, and referenced by numbers in
% brackets~\cite{ethics, Klemmer:2002:WSC:503376.503378,
%   Mather:2000:MUT, Zellweger:2001:FAO:504216.504224}. For papers from
% conference proceedings, include the title of the paper and an
% abbreviated name of the conference (e.g., for Interact 2003
% proceedings, use \textit{Proc. Interact 2003}). Do not include the
% location of the conference or the exact date; do include the page
% numbers if available. See the examples of citations at the end of this
% document. Within this template file, use the \texttt{References} style
% for the text of your citation.

% Your references should be published materials accessible to the
% public.  Internal technical reports may be cited only if they are
% easily accessible (i.e., you provide the address for obtaining the
% report within your citation) and may be obtained by any reader for a
% nominal fee.  Proprietary information may not be cited. Private
% communications should be acknowledged in the main text, not referenced
% (e.g., ``[Robertson, personal communication]'').

\begin{table}
  \centering
  \begin{tabular}{l r r r}
    % \toprule
    & & \multicolumn{2}{c}{\small{\textbf{Test Conditions}}} \\
    \cmidrule(r){3-4}
    {\small\textit{Name}}
    & {\small \textit{First}}
      & {\small \textit{Second}}
    & {\small \textit{Final}} \\
    \midrule
    Marsden & 223.0 & 44 & 432,321 \\
    Nass & 22.2 & 16 & 234,333 \\
    Borriello & 22.9 & 11 & 93,123 \\
    Karat & 34.9 & 2200 & 103,322 \\
    % \bottomrule
  \end{tabular}
  \caption{Table captions should be placed below the table. We
    recommend table lines be 1 point, 25\% black. Minimize use of
    table grid lines.}~\label{tab:table1}
\end{table}

\section{Sections}

The heading of a section should be in Helvetica or Arial 9-point bold,
all in capitals. Sections should \textit{not} be numbered.

\subsection{Subsections}

Headings of subsections should be in Helvetica or Arial 9-point bold
with initial letters capitalized.  For sub-sections and
sub-subsections, a word like \emph{the} or \emph{of} is not
capitalized unless it is the first word of the heading.

\subsubsection{Sub-subsections}

Headings for sub-subsections should be in Helvetica or Arial 9-point
italic with initial letters capitalized.  Standard
\texttt{{\textbackslash}section}, \texttt{{\textbackslash}subsection},
and \texttt{{\textbackslash}subsubsection} commands will work fine in
this template.

\section{Figures/Captions}

Place figures and tables at the top or bottom of the appropriate
column or columns, on the same page as the relevant text (see
Figure~\ref{fig:figure1}). A figure or table may extend across both
columns to a maximum width of 17.78 cm (7 in.).

\begin{figure*}
  \centering
  \includegraphics[width=1.75\columnwidth]{figures/map}
  \caption{In this image, the map maximizes use of space. You can make
    figures as wide as you need, up to a maximum of the full width of
    both columns. Note that \LaTeX\ tends to render large figures on a
    dedicated page. Image: \ccbynd~ayman on
    Flickr.}~\label{fig:figure2}
\end{figure*}

Captions should be Times New Roman or Times Roman 9-point bold.  They
should be numbered (e.g., ``Table~\ref{tab:table1}'' or
``Figure~\ref{fig:figure1}''), centered and placed beneath the figure
or table.  Please note that the words ``Figure'' and ``Table'' should
be spelled out (e.g., ``Figure'' rather than ``Fig.'') wherever they
occur. Figures, like Figure~\ref{fig:figure2}, may span columns and
all figures should also include alt text for improved accessibility.
Papers and notes may use color figures, which are included in the page
limit; the figures must be usable when printed in black-and-white in
the proceedings.

The paper may be accompanied by a short video figure up to five
minutes in length. However, the paper should stand on its own without
the video figure, as the video may not be available to everyone who
reads the paper.  

\subsection{Inserting Images}
When possible, include a vector formatted graphic (i.e. PDF or EPS).
When including bitmaps,  use an image editing tool to resize the image
at the appropriate printing resolution (usually 300 dpi).

\section{Quotations}
Quotations may be italicized when \textit{``placed inline''} (Anab,
23F).

\begin{quote}
Longer quotes, when placed in their own paragraph, need not be
italicized or in quotation marks when indented (Ramon, 39M).  
\end{quote}

\section{Language, Style, and Content}

The written and spoken language of SIGCHI is English. Spelling and
punctuation may use any dialect of English (e.g., British, Canadian,
US, etc.) provided this is done consis- tently. Hyphenation is
optional. To ensure suitability for an international audience, please
pay attention to the following:

\begin{itemize}
\item Write in a straightforward style.
\item Try to avoid long or complex sentence structures.
\item Briefly define or explain all technical terms that may be
  unfamiliar to readers.
\item Explain all acronyms the first time they are used in your
  text---e.g., ``Digital Signal Processing (DSP)''.
\item Explain local references (e.g., not everyone knows all city
  names in a particular country).
\item Explain ``insider'' comments. Ensure that your whole audience
  understands any reference whose meaning you do not describe (e.g.,
  do not assume that everyone has used a Macintosh or a particular
  application).
\item Explain colloquial language and puns. Understanding phrases like
  ``red herring'' may require a local knowledge of English.  Humor and
  irony are difficult to translate.
\item Use unambiguous forms for culturally localized concepts, such as
  times, dates, currencies, and numbers (e.g., ``1--5--97'' or
  ``5/1/97'' may mean 5 January or 1 May, and ``seven o'clock'' may
  mean 7:00 am or 19:00). For currencies, indicate equivalences:
  ``Participants were paid {\fontfamily{txr}\selectfont \textwon}
  25,000, or roughly US \$22.''
\item Be careful with the use of gender-specific pronouns (he, she)
  and other gendered words (chairman, manpower, man-months). Use
  inclusive language that is gender-neutral (e.g., she or he, they,
  s/he, chair, staff, staff-hours, person-years). See the
  \textit{Guidelines for Bias-Free Writing} for further advice and
  examples regarding gender and other personal
  attributes~\cite{Schwartz:1995:GBF}. Be particularly aware of
  considerations around writing about people with disabilities.
\item If possible, use the full (extended) alphabetic character set
  for names of persons, institutions, and places (e.g.,
  Gr{\o}nb{\ae}k, Lafreni\'ere, S\'anchez, Nguy{\~{\^{e}}}n,
  Universit{\"a}t, Wei{\ss}enbach, Z{\"u}llighoven, \r{A}rhus, etc.).
  These characters are already included in most versions and variants
  of Times, Helvetica, and Arial fonts.
\end{itemize}

\section{Accessibility}
The Executive Council of SIGCHI has committed to making SIGCHI
conferences more inclusive for researchers, practitioners, and
educators with disabilities. As a part of this goal, the all authors
are asked to work on improving the accessibility of their
submissions. Specifically, we encourage authors to carry out the
following five steps:
\begin{enumerate}
\item Add alternative text to all figures
\item Mark table headings
\item Add tags to the PDF
\item Verify the default language
\item Set the tab order to ``Use Document Structure''
\end{enumerate}
For more information and links to instructions and resources, please
see: \url{http://chi2016.acm.org/accessibility}.  The
\texttt{{\textbackslash}hyperref} package allows you to create well tagged PDF files,
please see the preamble of this template for an example.

\section{Page Numbering, Headers and Footers}
Your final submission should not contain footer or header information
at the top or bottom of each page. Specifically, your final submission
should not include page numbers. Initial submissions may include page
numbers, but these must be removed for camera-ready. Page numbers will
be added to the PDF when the proceedings are assembled.

\section{Producing and Testing PDF Files}

We recommend that you produce a PDF version of your submission well
before the final deadline.  Your PDF file must be ACM DL
Compliant. The requirements for an ACM Compliant PDF are available at:
{\url{http://www.sheridanprinting.com/typedept/ACM-distilling-settings.htm}}.

Test your PDF file by viewing or printing it with the same software we
will use when we receive it, Adobe Acrobat Reader Version 10. This is
widely available at no cost. Note that most
reviewers will use a North American/European version of Acrobat
reader, so please check your PDF accordingly.

When creating your PDF from Word, ensure that you generate a tagged
PDF from improved accessibility. This can be done by using the Adobe
PDF add-in, also called PDFMaker. Select Acrobat | Preferences from
the ribbon and ensure that ``Enable Accessibility and Reflow with
tagged Adobe PDF'' is selected. You can then generate a tagged PDF by
selecting ``Create PDF'' from the Acrobat ribbon.

\section{Conclusion}

It is important that you write for the SIGCHI audience. Please read
previous years' proceedings to understand the writing style and
conventions that successful authors have used. It is particularly
important that you state clearly what you have done, not merely what
you plan to do, and explain how your work is different from previously
published work, i.e., the unique contribution that your work makes to
the field. Please consider what the reader will learn from your
submission, and how they will find your work useful. If you write with
these questions in mind, your work is more likely to be successful,
both in being accepted into the conference, and in influencing the
work of our field.

\section{Acknowledgments}

Sample text: We thank all the volunteers, and all publications support
and staff, who wrote and provided helpful comments on previous
versions of this document. Authors 1, 2, and 3 gratefully acknowledge
the grant from NSF (\#1234--2012--ABC). \textit{This whole paragraph is
  just an example.}

% Balancing columns in a ref list is a bit of a pain because you
% either use a hack like flushend or balance, or manually insert
% a column break.  http://www.tex.ac.uk/cgi-bin/texfaq2html?label=balance
% multicols doesn't work because we're already in two-column mode,
% and flushend isn't awesome, so I choose balance.  See this
% for more info: http://cs.brown.edu/system/software/latex/doc/balance.pdf
%
% Note that in a perfect world balance wants to be in the first
% column of the last page.
%
% If balance doesn't work for you, you can remove that and
% hard-code a column break into the bbl file right before you
% submit:
%
% http://stackoverflow.com/questions/2149854/how-to-manually-equalize-columns-
% in-an-ieee-paper-if-using-bibtex
%
% Or, just remove \balance and give up on balancing the last page.
%
\balance{}

\section{References Format}
Your references should be published materials accessible to the
public. Internal technical reports may be cited only if they are
easily accessible and may be obtained by any reader for a nominal
fee. Proprietary information may not be cited. Private communications
should be acknowledged in the main text, not referenced (e.g.,
[Golovchinsky, personal communication]). References must be the same
font size as other body text. References should be in alphabetical
order by last name of first author. Use a numbered list of references
at the end of the article, ordered alphabetically by last name of
first author, and referenced by numbers in brackets. For papers from
conference proceedings, include the title of the paper and the name of
the conference. Do not include the location of the conference or the
exact date; do include the page numbers if available. 

References should be in ACM citation format:
\url{http://www.acm.org/publications/submissions/latex_style}.  This
includes citations to Internet
resources~\cite{CHINOSAUR:venue,cavender:writing,psy:gangnam}
according to ACM format, although it is often appropriate to include
URLs directly in the text, as above. Example reference formatting for
individual journal articles~\cite{ethics}, articles in conference
proceedings~\cite{Klemmer:2002:WSC:503376.503378},
books~\cite{Schwartz:1995:GBF}, theses~\cite{sutherland:sketchpad},
book chapters~\cite{winner:politics}, an entire journal
issue~\cite{kaye:puc},
websites~\cite{acm_categories,cavender:writing},
tweets~\cite{CHINOSAUR:venue}, patents~\cite{heilig:sensorama}, 
games~\cite{supermetroid:snes}, and
online videos~\cite{psy:gangnam} is given here.  See the examples of
citations at the end of this document and in the accompanying
\texttt{BibTeX} document. This formatting is a edited version of the
format automatically generated by the ACM Digital Library
(\url{http://dl.acm.org}) as ``ACM Ref.'' DOI and/or URL links are
optional but encouraged as are full first names. Note that the
Hyperlink style used throughout this document uses blue links;
however, URLs in the references section may optionally appear in
black.

% BALANCE COLUMNS
\balance{}

% REFERENCES FORMAT
% References must be the same font size as other body text.
\bibliographystyle{SIGCHI-Reference-Format}
\bibliography{sample}

\end{document}

%%% Local Variables:
%%% mode: latex
%%% TeX-master: t
%%% End:
